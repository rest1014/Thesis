%%==========================================================================================
%%
%% This is file 'FBMStyle.tex'  v0.98
%% see also file 'FBMStyle_develop.tex
%%
%% Beta-Version eines Styles von Michael Arnemann
%% basieren auf einem Muster von Uwe Kappler 
%% 060905ar
%%========================================================================================== 
\documentclass[
        final,                % wenn es denn fertig ist.. 
        12pt,                 % Schriftgr��e (10pt, 11pt, 12pt) 11:pala:ok, 
        twoside,              % oneside, twoside
        pointlessnumbers,     % Kapitelnummer ohne . am Ende
        normalheadings,       % Groesse der Ueberschrift (bigheadings, normalheadings, 
        halfparskip,          % Europ�ischer Satz mit Abstand zwischen Abs�tzen
        idxtotoc,             % Index ins Verzeichnis einf�gen	*** pr�fen      
        headsepline,          % Strich unter Kopfzeile
        headinclude,          % Kopfzeile bei Satzspiegel bereucksichtigen
        DIV13,                % mit palatino 11pt oder CM 12
        BCOR14mm]             % Binderand    %1mm, f�r Klebebindung % 14mm dann etwa gleichm��ig
   {scrbook}                  % Dokumenttyp // f�r Dipomarbeiten, Diss
%----------------------------------------------------------------------------------------
%  PACKAGES
%----------------------------------------------------------------------------------------
\typearea                  % nach der Schrift
        [current]          % Heftrand BCOR oben definiert
%        {calc}             % DIV neu berechnen aus package 
        {last}             % DIV letzte aus Definition zuvor

%\usepackage{scrpage2}        % Package laden 
\usepackage[light,math]{iwona}
\usepackage[T1]{fontenc}     % T1-encoded fonts: auch W�rter mit Umlauten trennen
\usepackage[T1]{url}         % much like \verb allow line breaks for paths and URLs
\usepackage[latin1]{inputenc} % Deutsche Umlaute: Eingabe nach ISO 8859-1 (Latin1)
\usepackage{epsfig,xspace}    % PS Bilder
\usepackage{graphicx,color}   % JPEG und PNG 
\usepackage{subfigure}	    % Mehrere Bilder in einem mit ver. Bildunterschriften
\usepackage{wrapfig}	    % Text um Bild

\usepackage[]{hyperref}         
\usepackage{ngerman}          % Neue deutsche Rechtschreibung
\usepackage{latexsym}         % Sonderzeichen
\usepackage{pifont}           % dito
\usepackage{array}            % f�r aufw�ndigere Tabellen
\usepackage{longtable}        % seiten�bergreifende Tabellen passt zu KOMA
\usepackage{multicol}         % Mehrspaltiger Satz
\usepackage{makeidx}          % f�r Index-Erstellung 
\usepackage{listings}         % f�r Latex Quelltext
\usepackage{textcomp}        % for upright mu (\textmu)
\usepackage[german=swiss]{csquotes}  % f�r quotes
\usepackage[fleqn]{amsmath}
\usepackage{amssymb}
\usepackage{eqnarray}	    % nummerierte und unnummerierte Gleichungen/systeme
\usepackage{ifthen}
\usepackage{fancybox} % f�r schattierte ovale Boxen etc. geht nicht mit Miktex 5 060614ar
%----------------------------------------------------------------------
% Paket um Listings sauber zu formatieren.
% SUPER !
%----------------------------------------------------------------------
\usepackage{listings}
% ---------------------------------------------------------------------------
% Listing Definationen f�r PHP Code
% ---------------------------------------------------------------------------
\definecolor{lbcolor}{rgb}{0.85,0.85,0.85}
\lstset{language=[LaTeX]TeX,
	numbers=left,
	stepnumber=1,
	numbersep=5pt,
	numberstyle=\tiny,
	breaklines=true,
	breakautoindent=true,
	postbreak=\space,
	tabsize=2,
	basicstyle=\ttfamily\footnotesize,
	showspaces=false,
	showstringspaces=false,
	extendedchars=true,
	backgroundcolor=\color{lbcolor}}

% ---------------------------------------------------------------------------
% Neue Umgebungen
% ---------------------------------------------------------------------------
\newenvironment{ListChanges}%
	{\begin{list}{$\diamondsuit$}{}}%
	{\end{list}}
%
%% **** END OF CLASS MStyle ****