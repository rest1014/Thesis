%%
%%
%%
\thispagestyle{empty}
\chapter*{Kurzfassung}
%
{ \itshape
Muster f�r eine Kurzfassung, maximal eine Seite }\\


Axiall�fter finden weite Verbreitung, z. B. zur K�hlung von Kraftfahrzeugmotoren. Im Hinblick auf den Energieverbrauch soll der L�fterwirkungsgrad m�glichst hoch sein. Ein wichtiger Parameter zur Optimierung des Wirkungsgrades ist die Drallverteilung l�ngs des Radius.

Die Aufgabe bei der vorliegenden Arbeit bestand darin, eine bez�glich des Wirkungsgrades m�glichst g�nstige Drallverteilung bei typischen Axiall�ftern zur K�hlung von Kraftfahrzeugmotoren zu finden. 

Zur L�sung der gestellten Aufgabe wurden nach sorgf�ltiger Versuchsplanung zwei 
Laufradvarianten nach dem Verfahren von C. M�ller ausgelegt, 
in der Versuchswerkstatt gefertigt und auf dem L�fter-Normpr�fstand im Labor XY bei der Firma ABC untersucht:
\bi
	\item Variante A mit konstantem Drall,
	\item Variante B mit nach au�en linear zunehmendem Drall.
\ei
Als Ergebnis der Untersuchungen zeigt sich, dass die Schaufeln der Variante A r�umlich stark verwunden sind. Variante B besitzt wenig verwundene, leichter zu fertigende Schaufeln. Die Kennlinie der Variante A ist instabil, die der Variante B ist stabil, was sich insbesondere bei F�rderung gegen h�heren Druck g�nstig auswirkt. Der Bestwirkungsgrad der Variante B ist mit 55 \% aber schlechter als der der Variante A (60 \%). Die aufgrund der Rechnung zu erwartende Beeinflussung des Nachstroms hinter den Laufr�dern ist bei beiden Varianten nicht beobachtbar.

Aus den Untersuchungen ist zu vermuten, dass eine hyperbolische Drallverteilung l�ngs des Radius den Nachteil des schlechteren Bestwirkungsgrades der Variante B nicht aufweist, sonst aber alle ihre Vorteile beibeh�lt. Weitere Untersuchungen zur Best�tigung dieser Vermutung sind n�tig.


\cleardoublepage