%%
%% Kapitel:
%%
\chapter{Referenzen}
\label{cha:Referenzen}
%%======================================================================



%----------------------------------------------------------------
\section{Referenzen}

Die Kapitel, Abschnitte, Gleichungen, Bilder, Tabellen etc.
werden mit \verb#\label{name}# benannt.

Die Referenz wird gebildet indem \verb#\ref{name}# im Text erscheint.
soll auf die Seite verwiesen werden, auf der dieses Objekt steht, wird
der Befehl \verb#\pageref{name}# genutzt.\\
Beispiel: Das Kapitel \emph{Listen} hat die Nummer \ref{sec:Listen}
und beginnt auf Seite \pageref{sec:Listen}.

\textbf{Wichtig: Gro�/Kleinschreibung wird unterschieden!}\\

Tabelle \emph{Tabellen�berschrift008} hat die Nummer \ref{tbl:Mit_allem} und
steht auf Seite \pageref{tbl:Mit_allem}.


Die Einsteinsche Gleichung steht auf Seite \pageref{eqt:einstein}
und hat die Nummer \ref{eqt:einstein}.

Das Bild steht auf Seite \pageref{fig:bild1}
und hat die Nummer \ref{fig:bild1}.


%----------------------------------------------------------------
\section{Fu�noten}

Ein Beispiel f�r eine Fu�note\footnotemark. 
\footnotetext{Und hier ist sie, die Fu�note!}


%----------------------------------------------------------------
\section{Zitate}

Literatur wird mit dem Befehl \verb#\cite{name}# zitiert, 
\zB \cite{Einstein}.

%
%
%\input{080_Macros/fuellung}
%
%