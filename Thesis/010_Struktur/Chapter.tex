%%
%% Kapitel:
%%
\chapter{Einteilung in Kapitel}
\label{cha:Einteilung_in_Kapitel}      %% Keine Umlaute! Keine Leerzeichen!

%\index{Einteilung}\index{Kapitel@\texttt{\textbackslash Kapitel}|textbf}
%%======================================================================

Das hier ist eine Datei, die ein einzelnes Kapitel enth�lt. Das
ganze Buch wird aus solchen Dateien zusammengesetzt, in dem man
mittels einer Include Anweisung in MAIN.TEX die einzelnen Kapitel
hintereinander h�ngt. 

\section{Ein Unterkapitel}
Nach dem Chapter kommen die Sections und dann etwas roter Text gefolgt von
unsichtbar (weiss) formatiertem Text. Dann wieder rot und schlie�lich 
verstecker Text.

\textcolor{red}{rot geschrieben}
ausgeblendet \textcolor{white}{weiss geschrieben}
\textcolor{red}{rot geschrieben}
N�chste Zeile versteckt.
\Hide{1 Franz jagt im komplett verwahrlosten Taxi quer durch Bayern.}\\
2. Franz jagt im komplett verwahrlosten Taxi quer durch Bayern.
Franz jagt im komplett verwahrlosten Taxi quer durch Bayern.
Franz tastet die Taste \Ovalbox{strg}.
Franz jagt im komplett verwahrlosten Taxi quer durch Bayern.


%------------------------------------------------------------------------------
\section{Abschnitte}

Nat�rlich kann l�sst sich der Text noch weiter unterteilen. Die n�chste Stufe
ist dann ein Unterabschnitt oder auch: {\verb# \subsection #}.

\subsection{Unterabschnitte}
Hier wird gezeigt wie Source-Code darstellt werden kann. Die Einfachheit erschlie�t sich dem geneigten Betrachter spontan, jedoch erst im Quelltext der \LaTeX-Datei.
%
%---- 
%
\lstinputlisting[caption=iMain.tex, label=lst:beispiel01, frame=tb]%
	{_iMain.tex}


%------------------------------------------------------------------------------

%======================================================================================
\section{Acrobat und TeXnicCenter}

Hier ein paar eingerahmte Hinweise zum Umgang mit dem TeXnicCenter und MiKTeX.

\begin{lstlisting}

Using Acrobat reader
TeXnicCenter has the ability to integrate nicely with Adobe Acrobat, such that recently compiled PDF files are opened automatically, and existing open PDF files are closed automatically upon recompilation. This is done using DDE calls to Acrobat.

You can insert the following lines of command in TeXnicCenter to integrate it with Acrobat. This particularly enables you to compile your files in pdf without having to close the previous document first (necessary in the default option). Integration is setup in the Build | Define Output Profiles menu option. Select a profile "LaTeX=>PDF" or "LaTeX=>PS=>PDF", and click on the viewer window :

view project's output:  [v] DDE command.\\
command: [DocOpen("%bm.pdf")][FileOpen("%bm.pdf")]\\
server: acroview\\
topic: control\\

forward search: [v] DDE command.\\
command: [DocOpen("%bm.pdf")][FileOpen("%bm.pdf")]\\
server: acroview\\
topic: control\\

Close document before running (La)TeX: [v] DDE command\\
command: [DocClose("%bm.pdf")]\\
server: acroview\\
topic: control\\

%\end{verbatim}
\end{lstlisting}




