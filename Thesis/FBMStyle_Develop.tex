%%==========================================================================================
%%
%% This is file 'FBMStyle_develop.tex'  v0.99
%%
%% Beta-Version eines Styles von M. Arnemann
%% basieren auf einem Muster von Herrn Kappler 
%% 060905ar
%% 060912ar
%%========================================================================================== 
\documentclass[
%        draft,                % Bilder nur mit Rahmen,  gut zur TextFehlersuche im dvi !!!!
        final,                % wenn es denn fertig ist.. 
        12pt,                 % Schriftgr��e (10pt, 11pt, 12pt) 11:pala, 10:Helv:ok, 
%        oneside,              % oneside f�r Draft 
        twoside,              % twoside
        onecolumn,            % Spalten (onecolumn/twocolumn)
%        twocolumn,
        pointlessnumbers,     % Kapitelnummer ohne . am Ende
        normalheadings,       % Groesse der Ueberschrift (bigheadings, normalheadings, smallheadings)
% Abs�tze
         halfparskip,         % Europ�ischer Satz mit Abstand zwischen Abs�tzen
%      
        idxtotoc,             % Index ins Verzeichnis einf�gen	*** pr�fen      
% Striche
%        headsepline,          % Strich unter Kopfzeile
        headinclude,          % Kopfzeile bei Satzspiegel bereucksichtigen
%        plainheadsepline,     % Strich auch beim Kapitelanfang (nicht ueblich)
%        footsepline,          % Strich ueber Fusszeile
%        footinclude,          % Fusszeile bei Satzspiegel beruecksichtigen
%        plainfootsepline,     % Strich auch beim Kapitelanfang
% Papier A4 (=Vorgabe) 
%        a4paper,              % Papier    
% Satzspiegel        
%        DIVcalc,              % Dichte des Druckes
%        DIVclassic,
%        DIV13,                % 8-15 f�r 11pt:std:10;  12pt:std:12, h�ngt vom Font ab !
                               % 13: mit palatino 11pt oder CM 12
         DIV15,                      
%        DIV15,                % mit palatino  12pt und Zeilenabstand > 1.2
        BCOR14mm ]            % Binderand  % 1mm, f�r Klebebindung % 14mm dann etwa gleichm��ig (Lochung)                                  % *** ! in der Datei f�r Titel ber�cksichtigen ! 
        {scrbook}             % Dokumenttyp // f�r Dipomarbeiten, Diss
%       {scrreprt}            % Dokumenttyp // Bericht bis etwa 80 Seiten
%       {scrartcl}            % Dokumenttyp // Seminararbeien 30 Seiten


%==============================================================================
%  PACKAGES
%==============================================================================
%\usepackage{calc}             % Rechnen innerhalb von Latex


%%-----------------------------------------------------------------------------
%% FONTS *** wird nichts aktiviert, dann gilt: Font Computer Modern ********
%       weitere Fonts: siehe online Help vom TechnicCenter: Font adttributes
%%-----------------------------------------------------------------------------

% - Ohne Serifen -           	% geht nicht
%\usepackage[scaled=.90]{helvet}	% etwas kleiner skaliert, damit es zu den anderen passt
%\usepackage{avant}          	% geht nicht
%\usepackage{garamond}
%
%% - Mit Serifen -
%\usepackage{times}          	% schmal
%\usepackage{mathptmx}       	% Times + passende Mathefonts

% *** den Vorschl�gen aus KOMA-Skript entsprechend, passt dieser Font
%     f�r die gew�hlte Textbreite am besten 
\usepackage{palatino}       	% breiter              **** sieht ganz gut aus 
%\usepackage{mathpazo}       	% passend zu palatino  **** 

%\usepackage{bookman}        	% breiter als palatino! 

%\usepackage{newcent}        	% for sophisticated font style
%\usepackage{lucid}          	% geht nicht
%
% -- Font Wahl (alternativ)
%
%\renewcommand{\familydefault}{phv}  % Adobe Helvetica 
%\renewcommand{\familydefault}{pag}  % AvantGarde-Book 
%\renewcommand{\familydefault}{ptm}  % Adobe Times (ziemlich eng)
%\renewcommand{\familydefault}{ppl}  % Palatino     (weiter) 
%\renewcommand{\familydefault}{pcr}  % Adobe Courier
%
% ** KOMA FONTS ** Schriften im Komaskript umdefinieren
%
%\setkomafont{chapter}{\rmfamily\huge}               % Chapter
%\setkomafont{section}{\rmfamily\Large}              % Section
%\setkomafont{subsection}{\rmfamily\large}           % subsection
%\setkomafont{subsection}{\rmfamily\normalsize}      % subsection
%
%\setkomafont{chapter}{\sffamily\huge}               % Chapter
%\setkomafont{section}{\sffamily\Large}              % Section
%\setkomafont{subsection}{\sffamily\large}           % subsection
%\setkomafont{subsection}{\sffamily\normalsize}      % subsection
%\setkomafont{pagehead}{\small\sffamily\slshape}     % Kopfzeile
%\setkomafont{pagenumber}{\bfseries\sffamily}        % Seitenzahl
%\setkomafont{sectioning}{\sffamily}                 % Titelzeilen
%\setkomafont{captionlabel}{\sffamily\bfseries\small}% Schrift f�r 'Abbildung' usw.
%\setkomafont{descriptionlabel}{\normalfont\bfseries}
%\addtokomafont{caption}{\sffamily\small\raggedright}% kleinere Schrift, linksb�ndig

\setkomafont{footnote}{\normalsize}                     % Marke und Text einer Fu�note
%\setkomafont{footnotelabel}{\normalsize}            %  
%\setkomafont{footnotereference}{\large}             % im Text
%
%%----------------------------------------------------------------------
%% Sprache
%%----------------------------------------------------------------------
%
\usepackage[T1]{fontenc}      % T1-encoded fonts: auch W�rter mit Umlauten trennen
\usepackage[T1]{url}          % much like \verb allow line breaks for paths and URLs
%\usepackage[OT1]{fontenc}    % Deutsche Umlaute im Text, klappt nicht immer gut
\usepackage[latin1]{inputenc} % Deutsche Umlaute: Eingabe nach ISO 8859-1 (Latin1)
                              % siehe Emacs: iso-accents-mode
%\usepackage[utf8]{inputenc}  % Eingabe nach UTF-8
%\usepackage{ae}              % almost european, virtueller T1-Font
\usepackage{ngerman}          % Neue deutsche Rechtschreibung

%%----------------------------------------------------------------------
%% Verschiedenes
%%----------------------------------------------------------------------
\usepackage[]{hyperref}         

\usepackage{latexsym}         		% Sonderzeichen
\usepackage{pifont}           		% dito
\usepackage{array}            		% f�r aufw�ndigere Tabellen
\usepackage{textcomp}        		% for upright mu (\textmu)
\usepackage[german=swiss]{csquotes}  % f�r quotes

%%----------------------------------------------------------------------
%% Satz (Seitenlayout)
%%----------------------------------------------------------------------
% -- Zeilenabstand vor typearea setzen! damit ganzzahlige Anzahl von Zeilen pro Seite
\usepackage{setspace}      	% Setzt Zeilenabstand  %\onehalfspacing
%\doublespace	         	% 2-facher Abstand
%\onehalfspace              	% 1,5-facher Abstand
\typearea                  	% nach der Schrift
        [current]          	% Heftrand BCOR oben definiert
%        {calc}             	% DIV neu berechnen aus package 
        {last}             	% DIV letzte aus Definition zuvor

\usepackage{multicol}      	% Mehrspaltiger Satz

%%----------------------------------------------------------------------
%% Kopf- und Fusszeilen
%%----------------------------------------------------------------------
\usepackage{scrpage2}          
%\pagestyle{scrheadings}            % Kopf- und Fusszeilen
%\clearscrheadfoot                  % Alles auf "" setzen

%% In [] steht, was am Kaptelanfang angezeigt wird, in {} steht,
%% was auf den uebrigen Seiten angezeigt wird. 
%% headmark = aktuelle Ueberschrift
%% pagemark = aktuelle Seite


%%%%% Kopf
%% ihead = Kopfzeile innen (links bei einseitigem Layout)
%%\ihead[]{innen}
%\ihead[\pagemark]{\pagemark}

%% chead = Kopfzeile mittig
%%\chead[]{mittig}

%% ohead = Kopfzeile aussen (rechts bei einseitigem Layout)
%\ohead[]{\headmark}


%%%%% Fuss
%% ifoot = Fusszeile innen (links bei einseitigem Layout)
%%\ifoot[]{innen}

%% cfoot = Kopfzeile mittig
%%\cfoot[]{mittig}

%% ofoot = Fusszeile aussen (rechts bei einseitigem Layout)
%%\ofoot[\pagemark]{\pagemark}
%


%%----------------------------------------------------------------------
%% Bilder , Grafiken
%%----------------------------------------------------------------------
%
\usepackage{epsfig,xspace}    	% PS Bilder
\usepackage{graphicx,color}   	% JPEG und PNG 
\usepackage{subfigure}	    		% Mehrere Bilder in einem mit ver. Bildunterschriften
%\usepackage{wrapfig}	     	% Text um Bild
%  \begin{wrapfigure}[12]{r}[34pt]{5cm} <figure> \end{wrapfigure}
%  [number of narrow lines] {placement} [overhang] {width of figure}
%  Placement is one of   r, l, i, o, R, L, I, O,  for right, left,
%  inside, outside, (here / FLOAT).
%
% *** nicht ben�tigt
%\usepackage[DVIPS]{graphicx,color}   % JPEG und PNG  f�r DVI
%\usepackage[pdftex]{graphicx,color} % pr�fen ob geht
%\usepackage{epstopdf}
%\usepackage[final]{graphicx}  % um Graphiken einzubinden (Wirkung kl�ren)


%%----------------------------------------------------------------------
%% Tabellen
%%----------------------------------------------------------------------
\usepackage{longtable}        	% seiten�bergreifende Tabellen passt zu KOMA
%\usepackage{supertab}         	% mehrseitige Tabellen
%% farbige Hyperlinks im PDF

%\usepackage{colortbl}        	% farbige Tabellen (v. D. Carlisle)
\usepackage{makeidx}          	% f�r Index-Erstellung 
\usepackage{listings}         	% f�r Latex Quelltext

%
%%----------------------------------------------------------------------
% Glossary
%%----------------------------------------------------------------------
%%% geht (noch) nicht  060830ar
%\usepackage[style=long,         % 
%            header=none,        %
%            border=none,        %
%            number=none,        %
%            cols=2,             %
%            toc=true]{glossary} % f�r Glossary 
%\usepackage{glossary} 
%\renewcommand{\glossaryname}{Glossar} %Damit nicht "`Glossary"' erscheint
%\makeglossary 

%%----------------------------------------------------------------------
%    Mathematik
%%----------------------------------------------------------------------
\usepackage[fleqn]{amsmath}
\usepackage{amssymb}
\usepackage{eqnarray}	    % nummerierte und unnummerierte Gleichungen/systeme

%%----------------------------------------------------------------------
%    Sonstiges
%%----------------------------------------------------------------------
%\usepackage{bm}              % for boldmath
%\usepackage{cite}
%\usepackage{url}
%   \urlstyle{rm}% for List of Contributors
%        
%\usepackage{format-thm}% some predefined theorem environments

%\usepackage{ragged2e}
%\usepackage{endnotes}
%\usepackage{xspace}

\usepackage{ifthen}
%\usepackage{eso-pic}  %add picture commands (or backgrounds) to every page
%\usepackage{afterpage}
%\usepackage{relsize}
%
%--- im Main-doc definieren
%\usepackage{chapterthumb}    % f�r Skripte: ChapterNummer am Rand
%\pagestyle{scrheadings}
%\lohead[\putchapterthumb]{\putchapterthumb}
%\addtokomafont{chapterthumb}{\bfseries}
%%
%%
%% other interesting packages:
%%
% \usepackage{varioref}
% \usepackage{verbatim}
% \usepackage{subfigure}
\usepackage{fancybox} % f�r schattierte ovale Boxen etc. geht nicht mit Miktex 5 060614ar
  %Command  	Effect \command{content to be boxed} 
  % fbox 	      square box
  % shadowbox 	square box with shadow
  % doublebox 	double square box
  % ovalbox 		thin oval box
  % Ovalbox 		thick oval box
%
% \usepackage{tabularx} % automatische Spaltenbreite
% \usepackage{supertab} % mehrseitige Tabellen

%----------------------------------------------------------------------
%% Satzspiegel = bedruckter Bereich
%%----------------------------------------------------------------------
% \usepackage{typearea}
%
%% Manuelle Einstellung des bedruckbaren Bereichs (NUR IM NOTFALL!)
% \areaset
%         [20mm]                % Heftrand
%         {150mm}               % Breite bedruckbarer Bereich
%         {237mm}               % H�he dito
%%
%% KOMA l�sst gerne oben zu wenig Rand
%% hiermit kann man den oberen Rand verschieben
% \setlength{\topmargin}{+3mm}  % gut mit Strich �ber Fusszeile


%% Einzug der ersten Zeile eines Absatzes. Im englischen Sprachraum
%% ist es �blich, einen neuen Absatz durch Einzug zu kennzeichnen.
%% Wir tun das �blicherweise durch einen gr��eren Zeilendurchschu�
%
%\setlength{\parskip}{3pt}               % Abstand von Absatz zu Absatz
%\setlength{\parindent}{0pt}             % Einzug 1. Zeile
%\setlength{\mathindent}{1em minus 1em} % f�r fleqn


%%
%%
%%  siehe Diplomarbeit von  Diplomarbeit mit LaTeX
%   Copyright (c) 2002-2005  Tobias Erbsland, Andreas Nitsch
%\usepackage[savemem]{listings} % f�r Listings
%\lstloadlanguages{TeX}


%----------------------------------------------------------------------
% Paket um Listings sauber zu formatieren.
% SUPER !
%----------------------------------------------------------------------
%\usepackage[savemem]{listings}
\usepackage{listings}
%\lstloadlanguages{LaTeX,Scilab,Delphi,Fortran,Matlab}

% ---------------------------------------------------------------------------
% Listing Definationen f�r PHP Code
% ---------------------------------------------------------------------------
\definecolor{lbcolor}{rgb}{0.85,0.85,0.85}
\lstset{language=[LaTeX]TeX,
	numbers=left,
	stepnumber=1,
	numbersep=5pt,
	numberstyle=\tiny,
	breaklines=true,
	breakautoindent=true,
	postbreak=\space,
	tabsize=2,
	basicstyle=\ttfamily\footnotesize,
	showspaces=false,
	showstringspaces=false,
	extendedchars=true,
	backgroundcolor=\color{lbcolor}}

% ---------------------------------------------------------------------------
% footnote
% ---------------------------------------------------------------------------
%\deffootnote[1em]{1.5em}{1em}{\textsuperscript{\thefootnotemark}} % std
\deffootnote{1em}{1.5em}{\textsuperscript\thefootnotemark\ } %Text klebt nicht an Zahl

% ---------------------------------------------------------------------------
% Neue Umgebungen
% ---------------------------------------------------------------------------
\newenvironment{ListChanges}%
	{\begin{list}{$\diamondsuit$}{}}%
	{\end{list}}
%
%% **** END OF CLASS MStyle ****

