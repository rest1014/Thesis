%%
%% Kapitel:
%%
%% Status: 060813ar
%%
%%======================================================================

\chapter{Bilder}

Am Besten die Bilder liegen als eps- und(!) als pdf-, png-, oder jpg-Datei vor. 
Wenn keine Dateiendung im include-Befehl angegeben ist, wird automatisch die 
passende Datei geladen.


\section{Bilder im Format EPS}
\label{sec:BilderEPS}

Bilder mit dem Speicherformat eps werden empfohlen. Bestes Druckbild. Allerdings sind 
die Bilder meist ziemlich gro�.

\subsection{Erzeugung}
Wie k�nnen jpg-Dateien aus �blichen Anwendungen heraus erzeugt werden?
Das k�nnte hier gut beschrieben werden.

{\LaTeX} kann mit diesen Bildern sowohl dvi-Dateien als auch pdf-Dateien erzeugen.
YAP, der Viewer f�r dvi-Dateien, ist sehr n�tzlich beim Auffinden von Fehlern im Quelltext.
pdf-Dateien werden auf dem Umweg DVI-PS-PDF erzeugt.


%\section{Postscript}
%  
%Die einfachste Variante benutzt \emph{Postscript}. Hat den Vorteil,
%dass man CAD-Zeichnungen einfach einbinden kann. Beim Skalieren
%werden die Strichbreiten auch skaliert. JPGs kann man mit
%\texttt{jpeg2ps} in Postscript umwandeln.\\
%Nachteil: Drucker muss Postscript k�nnen, \texttt{ghostscript} muss installiert sein!
%{\begin{figure}[h]
%              \begin{center}%
%	      \includegraphics[width=0.8\textwidth]{001_Titel/hska_vor_cmyk.eps} 
%              \caption{\label{fig:bild}Postscriptbild}
%              \end{center}
%              \end{figure}
%             }






