%%
%% Kapitel:
%%
\chapter{Bilder mit PS}
\label{BilderPS}
%%======================================================================



\section{Bilder}

Etwas trickreich, da \LaTeX\ urspr�nglich daf�r nicht vorgesehen war.

\subsection{Postscript}
  
Die einfachste Variante benutzt \emph{Postscript}. Hat den Vorteil,
dass man CAD-Zeichnungen einfach einbinden kann. Beim Skalieren
werden die Strichbreiten auch skaliert. JPGs kann man mit
\texttt{jpeg2ps} in Postscript umwandeln.\\
Nachteil: Drucker muss Postscript k�nnen, \texttt{ghostscript} muss installiert sein!

{\begin{figure}[h]
              \begin{center}
              \epsfig{figure=0060/bild_ps.ps, width=100mm, angle=0}
              \caption{\label{fig:bild1ps} Postscriptbild} mit epsfig
              \end{center}
              \end{figure}
             }







