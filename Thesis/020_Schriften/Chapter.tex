%%
%% Kapitel:
%%
\chapter{Schriften}
\label{sec:Schriften}
%%======================================================================

Die Standardschriftfamilie wird bereits im Vorspann definiert
(siehe FBMStyle.tex). Wenn keine besonderen Vorgaben erfolgen, wird 
mit dem Standardfont gearbeitet (CM: Computer Modern). 
Bei dieser Auswahl sind die Fonts sehr gut aufeinander abgestimmt.
Liebhaber der
serifenlosen Schrift aktivieren \zB \\
\verb#\renewcommand{\familydefault}{phv}#.
Die Gleichungen werden allerdings weiter mit Serifen erscheinen.

\section{Schriftstil}
%
\textrm{Normalerweise ist der Text in diesem Schriftstil gesetzt.}
\textbf{Das hier ist fetter Text}. Jetzt kommt
\textsl{schr�ge Schrift} und hier kommt
\textit{eine andere Art schr�ger Schrift, die man kursiv nennt}.

 

\subsection{Hervorhebung}
%
In diesem Absatz ist ein Wort \emph{hervorgehoben}. Daf�r
gibt es den Befehl emph. \emph{Er hat den ganz gro�en 
Vorteil, dass man innerhalb einer Hervorhebung
etwas \emph{hervorheben} kann.} 


\section{Schriftart}
%
Das ist die normale proportionale Schriftart mit \emph{Serifen}.

\texttt{Das hier ist Typewriter, so ne Art\\ 
Schreibmaschinenschrift. 
Sie ist nichtproportional.
Es gibt sie auch in \textbf{fett} und
\textsl{schr�g}.}

\textsf{Das hier ist die proportionale Schrift ohne
Serifen. Gibts nat�rlich auch in \textsl{schr�g} und in \textbf{fett}
Wird normalerweise f�r �berschriften benutzt.}


\section{Schriftgr��en}
%
Die einzelnen Schriftgr��en werden in Abh�ngigkeit von
der Basisschrift automatisch berechnet. Aber manchmal
muss es

\tiny{winzig} 
\scriptsize{sehr klein} 
\footnotesize{ziemlich klein}
\small{klein}
\normalsize{normal}
\large{etwas gr��er} 
\Large{ziemlich gro�}
\LARGE{gro�}
\huge{sehr gro�}
\Huge{riesig}

\normalsize
