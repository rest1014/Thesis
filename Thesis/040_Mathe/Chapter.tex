%%
%% Kapitel:
%%
\chapter{Mathematik}
\label{cha:Mathematik}
%%======================================================================

\section{Im Text}
%
Zuerst ein paar griechische Buchstaben: $\alpha$, $\beta$, $\tau$, $\pi$ $\xi$ $\psi$ $\Psi$.
%
Und was der Ingenieur mag: $\sigma\idx{zul}$, $W\idx{xy}$. 


\section{Abgesetzte Gleichungen}
%
Gleichungen werden automatisch mit Nummer, die auf das Kapitel enth�lt versehen:
\begin{equation}
\label{eqt:einstein}
E = mc^2
\end{equation}

Variablen werden immer automatisch kursiv gedruckt. Das e f�r die Eulersche Zahl muss 
daher in Gleichungen extra  mit  {\verb# \mathrm{e} #} \newline
als Nicht-Variable gekennzeichnet werden.

Nach DIN stehen Integralgrenzen  ober- bzw. unterhalb.\newline
Dazu wird der Befehl  {\verb# \int\limits_0^3 #} genutzt. Ein Beispiel gibt
\RefEq{eqt:integral}
\begin{equation}
\label{eqt:integral}
   \int\limits_0^3{x^2}dx=9 .
\end{equation}
%
Indizes werden im Allgemeinen nicht kursiv geschrieben, dazu verhilft ein
eigenes Makro {\verb+\newcommand{\idx}[1]{_\mathrm{#1}+}. Ein Beispiel
wird in \RefEq{eqt:bauer1} gegeben
\begin{equation}
  \label{eqt:bauer1}
   \dot{Q}\idx{Fl"ache} \sim \frac{1}{A\idx{Aussenwandfassade}}.
\end{equation}
%

Gleichungen ohne Nummern k�nnen mit mit \verb# \begin{displaymath} bzw. \end # erzeugt werden, 
alternativ mit \verb# \[ \] #
\begin{displaymath}
\sum_1^{\Xi} {\Psi}^2 = \frac{
          \int_{a_i^3}^{\sqrt{\omega - z^7}}
      }{
          \sqrt{\sqrt{e^{x^-3}}}
     }.
\end{displaymath}
