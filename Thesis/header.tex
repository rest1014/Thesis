%%==========================================================================================
%% This is file 'FBMStyle.tex'
%%========================================================================================== 
\documentclass[
	pdftex,%              PDFTex verwenden
	a4paper,%             A4 Papier
	oneside,%             Einseitig
	bibtotocnumbered,%    Literaturverzeichnis nummeriert einf�gen
	idxtotoc,%            Index ins Verzeichnis einf�gen
	halfparskip,%        Europ�ischer Satz mit abstand zwischen Abs�tzen
	chapterprefix,%       Kapitel anschreiben als Kapitel
	headsepline,%         Linie nach Kopfzeile
	footsepline,%         Linie vor Fusszeile
	12pt%                 Gr�ssere Schrift, besser lesbar am bildschrim
]{scrbook}
%----------------------------------------------------------------------------------------
%  PACKAGES
%----------------------------------------------------------------------------------------
%
% Paket f�r die Indexerstellung.
%
\usepackage{makeidx}

%
% Paket f�r �bersetzungen ins Deutsche
%
\usepackage[french,ngerman]{babel}


%
% Paket f�r Quotes
%
\usepackage[babel,french=guillemets,german=swiss]{csquotes}

%
% Pakete um Latin1 Zeichnens�tze verwenden zu k�nnen und die dazu
%  passenden Schriften.
%
\usepackage[latin1]{inputenc}
\usepackage[T1]{fontenc}

%
% Paket um die Symbole des TS1 Zeichensatzes verwenden zu k�nnen.
%
%\usepackage{textcomp}

%
% Paket zum Erweitern der Tabelleneigenschaften
%
\usepackage{array}

%
% Paket um Grafiken einbetten zu k�nnen
%
%\usepackage{graphicx}

%
% Spezielle Schrift verwenden.
%
%\usepackage{goudysans}

%
% Spezielle Schrift im Koma-Script setzen.
%
\setkomafont{sectioning}{\normalfont\bfseries}
\setkomafont{captionlabel}{\normalfont\bfseries}
\setkomafont{pagehead}{\normalfont\itshape}
\setkomafont{descriptionlabel}{\normalfont\bfseries}

%
% Zeilenumbruch bei Bildbeschreibungen.
%
\setcapindent{1em}

%
% kopf und fusszeilen
%
\pagestyle{headings}

%
% mathematische symbole aus dem AMS Paket.
%
\usepackage{amsmath}
\usepackage{amssymb}

%
% Type 1 Fonts f�r bessere darstellung in PDF verwenden.
%
%\usepackage{mathptmx}           % Times + passende Mathefonts
%\usepackage[scaled=.92]{helvet} % skalierte Helvetica als \sfdefault
\usepackage{courier}            % Courier als \ttdefault

%
% Paket um Textteile drehen zu k�nnen
%

%
% Package f�r Farben im PDF
%
%\usepackage{color}
\usepackage{graphicx,color}   % JPEG und PNG 
	
	
	
% Paket f�r Links innerhalb des PDF Dokuments
%
%
% Index erzeucgen
%
\makeindex

%
% EOF
%
