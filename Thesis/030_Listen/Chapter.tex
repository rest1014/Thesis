%%
%% Kapitel:
%%
\chapter{Listen}
\label{sec:Listen}
%%======================================================================
%
Listen, Aufz�hlungen etc., pur oder auch gemischt.
%
\section{Spiegelstrich}
%
Eine Aufz�hlung mit Punkt davor sieht so aus:
%
\begin{itemize}
      \item Das ist ein Spiegelstrich.
      \item Das ist noch ein Spiegelstrich.
      \begin{itemize}
             \item Eine Stufe tiefer
             \item Verschachtelt    
      \end{itemize}
      \item Und wieder eine Stufe hoch.
\end{itemize}        


\section{Mit Nummern}
%
Eine Aufz�hlung mit Nummern davor sieht so aus:
\begin{enumerate}
      \item Das ist ein Spiegelstrich.
      \item Das ist noch ein Spiegelstrich.
      \begin{enumerate}
             \item Eine Stufe tiefer
             \item Verschachtelt 
             \begin{enumerate}
                   \item Noch weiter unten   
             \end{enumerate}
      \end{enumerate}
      \item Und wieder oben.
\end{enumerate} 



\section{Mit eigenen Items}

Eine Aufz�hlung mit eignen Beschriftungen:

\begin{description}
	\item[Erstes] Sie sitzen in der ersten Reihe, Sie sitzen in der ersten Reihe
	              Sie sitzen in der ersten Reihe, Sie sitzen in der ersten Reihe,
	              Sie sitzen in der ersten Reihe
	\item[Zweites] Sie sitzen in der ersten Reihe, Sie sitzen in der ersten Reihe,
								Sie sitzen in der ersten Reihe, Sie sitzen in der ersten Reihe,
								Sie sitzen in der ersten Reihe,
	\item[f�nfzehntes] Sie schalten ab, Sie schalten ab, Sie schalten ab,
	              Sie schalten ab, Sie schalten ab, Sie schalten ab.
\end{description}
%