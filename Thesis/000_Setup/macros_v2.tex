%===================================================================
% Makros       
%===================================================================
%
%
%%=============================================================================
%% Neue, neudefinierte Befehle
%%=============================================================================

%-------------------------------------------------------------------
% Abbildungen **** Empfehlung: Kopiervorlage nutzen und anpassen, nicht Makro!
%-------------------------------------------------------------------
% Vorsicht: Dateinamen in UNIX-Syntax und relativ zu main.tex
%
% Syntax gilt f�r alle unterst�tzten Grafikformate!
% Voraussetzung: die Datei existiert und die Endung wird NICHT 
% im Dateinamen angegeben!
%
% Syntax: 
% \fg {DATEINAME}% = caption
%          {Ort auf Blatt} htbf
%          {BREITE} % vorzugsweise als Teil der TExtbreite z.B. '0.8\textwidth'
%          {ROTATIONSWINKEL}%
%          {Bildtitel im Abbildungsverzeichnis}
%          {BILDUNTERSCHRIFT}%
%
% Kopiervorlage 
% \fg{Datei}{Ort}{Breite}{Rotation}{Bildtitel im Abbildungsverzeichnis}{Bildunterschrift}%

\newcommand{\fg}[6]
             {
              \begin{figure}[#1]
              \begin{center}
              \includegraphics[width=#3, angle=#4]{#2}
              \caption[#4]{%
              \label{fig:#2}%
              #5}
              \end{center}
              \end{figure}
             }

             
%-------------------------------------------------------------------
% Abbildungen caption beside
%-------------------------------------------------------------------
% Kopiervorlage: in den Quelltext kopieren, dann dort umbrechen
%\fgB{Datei}{Ort, !h oder tbh}{Breite}{Rotation}{Bildunterschrift}{i}{Text}
%
% Syntax: 
%\fgB {DATEINAME}
%        {Ort, !h oder tbh}   
%        {BREITE} % vorzugsweise als Teil der TExtbreite z.B. '0.8\textwidth'
%        {ROTATIONSWINKEL}
%        {BILDUNTERSCHRIFT}
%        {Ausrichtung Caption= i oder o}
%        {Text im Abbildungsverzeichnis}
%
% Vorsicht: Dateinamen in UNIX-Syntax und relativ zu main.tex
% evtl ist der Abstand zwischen Text und Bild zu klein
% dann �ndern (siehe Koma skript  
\newcommand{\fgB}[7]
             {
              \begin{figure}[#2]
              %\setcapindent*{1em}  %h�ngender Einzug ab 2. Zeile
              \begin{captionbeside}[#7]%
              {#5}[#6][\linewidth]%
              %[2em]* %Verschiebung nach aussen
              \includegraphics[width=#3, angle=#4]{#1}
              \end{captionbeside}%              
              \label{fig:#1}
              \end{figure}
             }             %-------------------------------------------------------------------
% Tabellen
%-------------------------------------------------------------------
% Syntax: \bt{SPALTENDEFINITION} - Begin Table
%         \et{LABEL}{TABELLENBEZEICHNUNG}
\newcommand{\bt}[1]
           { 
             \begin{table}[!h]
             \begin{center}
             \begin{tabular}{#1}
           }
\newcommand{\et}[2]
           {
            \end{tabular}
            \caption{\label{tbl:#1}#2}
            \end{center}
            \end{table}
           }

%-------------------------------------------------------------------
% Gleichungen 
%-------------------------------------------------------------------
% Syntax: \beq{NAME DER GLEICHUNG} 
%         \eeq
% Referenz: \ref{eqt:NAME DER GLEICHUNG}
%
% begin
\newcommand{\beq}[1]
           {
            \begin{equation}
            \label{eqt:#1}
           }
% end
\newcommand{\eeq}
           {
             \end{equation}
           }
%%--------------------------------------------------
%% Die Abteilung "Faulheit"
%%--------------------------------------------------
% Itemize 
\newcommand{\bi}{\begin{itemize}}
\newcommand{\ei}{\end{itemize}}

% Enumerate
\newcommand{\be}{\begin{enumerate}}
\newcommand{\ee}{\end{enumerate}}

% Fussnoten
% Syntax: \fn legt fest, wo das Fu�notenzeichen steht
%         \fnt{FUSSNOTENTEXT} legt den Text fest
\newcommand{\fn}{\footnotemark}
\newcommand{\fnt}[1]{\footnotetext{#1}}

%--------------------------------------------% Kurzbefehle <<<<<<<
\newcommand{\beqa}{\begin{eqnarray}}         
\newcommand{\eeqa}{\end{eqnarray}}

\newcommand{\ba}{\begin{array}}
\newcommand{\ea}{\end{array}}

\newcommand{\bdm}{\begin{displaymath}}
\newcommand{\edm}{\end{displaymath}}

\newcommand{\bd}{\begin{description}}
\newcommand{\ed}{\end{description}}

\newcommand{\MakeDateRight}
              {\begin{flushright} Hannover, den \heute\\ Ar\\
               \end{flushright}}        % Schreibt Ort, Datum, Verfasser rechts

\newcommand{\s}{\scriptscriptstyle}
\newcommand{\D}{\displaystyle}

\newcommand{\ol}{\ddot{O}l}
\newcommand{\p}{\partial}
\newcommand{\Dp}{\Delta p}
\newcommand{\te}{$\vartheta$}             % theta
\newcommand{\R}{{\em\bf R}}               % Universelle Gaskonstante fett
\newcommand{\C}{$^\circ$C}                % Grad Celsius
\newcommand{\mue}{\textmu}
\renewcommand{\d}{\partial\mspace{2mu}}   % partielles Diff. Zeichen 
\newcommand{\td}{\,\mathrm{d}}           	% totales Diff (d, nicht kursiv)
\newcommand{\ddt}[1]{\frac{\td #1}{\td t}}% zweifach 

\newcommand{\idx}[1]{_\mathrm{#1}}        % nicht kursiver Index in Gleichungen geht nur mit Umlauten wie "a


\newcommand{\bul}{$\bullet$}              % Mark in Tabs
\newcommand{\Q}{$\bullet$}                % mark2 in Tabs
\newcommand{\mc}{\multicolumn}


\def\zB{z.\,B.\ }
\def\dh{d.\,h.\ }
\def\ua{u.\,a.\ }
\def\su{s.\,u.\ }
\newcommand{\bzw}{bzw.\ }

%%
%% Refrenzen
%%
%\newcommand{\RefTab}[1]{$\underline{\mbox{Tab.~\ref{#1}}}$}   % 1. Tabellenref. im Text
%\newcommand{\RefFig}[1]{$\underline{\mbox{Abb.~\ref{#1}}}$}   % 1. Bildref.
\newcommand{\RefTab}[1]{\underline{Tabelle~\ref{#1}}}             % 1. Tabellenref. im Text
\newcommand{\RefFig}[1]{\underline{Abb.~\ref{#1}}}             % 1. Bildref.
\newcommand{\RefTabc}[1]{Tabelle~\ref{#1}}                        % 2. Tabellenref. im Text
\newcommand{\RefFigc}[1]{Abb.~\ref{#1}}                        % 2. Bildref.

\newcommand{\RefEq}[1]{\mbox{Gl.(\ref{#1})}}                  % Gleichungen

\newcommand{\ZmE}[2]{$#1$~{#2}}                                % Zahl:#1 mit Einheiten#2 #3
\newcommand{\EH}[2]{$#1$~{~#2}}                                % Zahl und Einheiten
\newcommand{\B}[1]{\mbox{#1}}

%-------------------------------------------------------------

\newfont{\ssf}{cmss10 scaled 1000}
\newfont{\ssb}{cmssbx10 scaled 1000}

\newcommand{\cf}{\ssf}                 % CaptionFonts: in Bildunter- ,Tab�berschriften
\newcommand{\rf}{\em}                  % RefFonts    : Kennzeichnung von Referenzen im Text
\newcommand{\eng}{\tt}                 % englische Begriffe im deutshcne Text 

% Hinweis
% Syntax: \oops{�BERSCHRIFT}{TEXT}
%        : nach �berschtift wird automatisch eingef�gt
\newcommand{\oops}[2]{\begin{quote}\textbf{#1}:\\ {#2} \end{quote} }


%---------------------------------------------------------------
% Bildunter- und Tabellen�berschriften
%---------------------------------------------------------------
% z.B. andere Schrift, oder auch Schriftform und andere Abk�rzung
%
% alternaiv : "Abbildungen" lassen und
%  Zeilenumbruch bei Bildbeschreibungen einf�hren \setcapindent{1em}
%
% bessere Methoden: siehe Komaskript
\def\figurename{Abb.}          % oder: Bild z.B. {\bfseries Abb.}
%\def\tablename{Tab.}          % oder: z.B. Tafel, Tab.
\renewcommand*{\captionformat}{.~} % DIN l�sst gr�ssen
\addtokomafont{caption}{\sffamily\small}% kleinere Schrift
\setkomafont{captionlabel}{\sffamily\bfseries}
\setcaphanging
\setcapindent{0em}             % kein Einzug


%---------------------------------------------------------------
% Literaturliste: Formatierung der Liste wird HIER vorgenommen!
%---------------------------------------------------------------
\newcommand{\lit}[4]
           {
              \bibitem{#1}
                 {#2:}
                 #3 
                 #4 
           }

%---------------------------------------------------------------
%
%---------------------------------------------------------------
% Kommentare 
\newcommand{\Kommentar}[1]{{\em #1}} 

% Alles innerhalb von \Hide{} oder \ignore{}
% wird von LaTeX komplett ignoriert (wie ein Kommentar)
\newcommand{\Hide}[1]{}
\let\ignore\Hide



