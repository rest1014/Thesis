%%
%% Kapitel:
%%
\chapter{Tabellen}
\label{cha:Tabellen}
%%======================================================================
%
\section{Einfach mit Tabs}
%
Das wichtigste zuerst: keine vertikalen Linien verwenden und nur wenig
horizontale! Das sieht besser aus.
%
\begin{tabbing}
\label{tbl:tabs}
%Tabulatoren setzen
Erste Spalte \hspace{7em} \= Zweite Spalte \hspace{7em} \= Dritte Spalte\\
%
eins \> zwei \> drei \\
eins \> zwei \> drei \\
eins \> zwei \> drei \\
%
\end{tabbing}
%
%----------------------------------------------------------------------
%
\section{Eine sch�ne Tabelle}
%
\begin{table}[!h]
\begin{center}
\caption{\label{tbl:Mit_allem}Tabellen�berschrift008}
\begin{tabular}{lrc}
 \hline \\
   linksb�ndig & rechtsb�ndig & zentriert \\
   \\
 \hline  \\
   123         &  123         & 123 \\
 
   eins        &  zwei        & drei \\
   \\
 \hline 

\end{tabular}

\end{center}
\end{table}

Wenn viele Tabellen erzeugt werden, sollte nach etwa 10 Tabellen ein
{\verb# \clearpage #} folgen. Das leert den Puffer und vermeidet Fehlermeldungen.


%%======================================================================
%%
%  test Tabellenverzeichnis
%

%-----------------------------------------------
\begin{table}
\begin{center}
\begin{tabular}{|l|r|c|}
 \hline
   linksb�ndig & rechtsb�ndig & zentriert \\
 \hline  
   123         &  123         & 123 \\
 \hline
   eins        &  zwei        & drei \\
 \hline 

\end{tabular}
\caption{Mit allem}
\end{center}
\end{table}

%-----------------------------------------------
\begin{table}
\begin{center}
\begin{tabular}{|l|r|c|}
 \hline
   linksb�ndig & rechtsb�ndig & zentriert \\
 \hline  
   123         &  123         & 123 \\
 \hline
   eins        &  zwei        & drei \\
 \hline 

\end{tabular}
\caption{Mit allem}
\end{center}
\end{table}
%-----------------------------------------------
\begin{table}
\begin{center}
\begin{tabular}{|l|r|c|}
 \hline
   linksb�ndig & rechtsb�ndig & zentriert \\
 \hline  
   123         &  123         & 123 \\
 \hline
   eins        &  zwei        & drei \\
 \hline 

\end{tabular}
\caption{Mit allem}
\end{center}
\end{table}

%-----------------------------------------------
\begin{table}
\begin{center}
\begin{tabular}{|l|r|c|}
 \hline
   linksb�ndig & rechtsb�ndig & zentriert \\
 \hline  
   123         &  123         & 123 \\
 \hline
   eins        &  zwei        & drei \\
 \hline 

\end{tabular}
\caption{Mit allem}
\end{center}
\end{table}
%-----------------------------------------------
\begin{table}
\begin{center}
\begin{tabular}{|l|r|c|}
 \hline
   linksb�ndig & rechtsb�ndig & zentriert \\
 \hline  
   123         &  123         & 123 \\
 \hline
   eins        &  zwei        & drei \\
 \hline 

\end{tabular}
\caption{Mit allem}
\end{center}
\end{table}

%-----------------------------------------------
\begin{table}
\begin{center}
\begin{tabular}{|l|r|c|}
 \hline
   linksb�ndig & rechtsb�ndig & zentriert \\
 \hline  
   123         &  123         & 123 \\
 \hline
   eins        &  zwei        & drei \\
 \hline 

\end{tabular}
\caption{Mit allem}
\end{center}
\end{table}
%-----------------------------------------------
\begin{table}
\begin{center}
\begin{tabular}{|l|r|c|}
 \hline
   linksb�ndig & rechtsb�ndig & zentriert \\
 \hline  
   123         &  123         & 123 \\
 \hline
   eins        &  zwei        & drei \\
 \hline 

\end{tabular}
\caption{Mit allem}
\end{center}
\end{table}

%-----------------------------------------------
\begin{table}
\begin{center}
\begin{tabular}{|l|r|c|}
 \hline
   linksb�ndig & rechtsb�ndig & zentriert \\
 \hline  
   123         &  123         & 123 \\
 \hline
   eins        &  zwei        & drei \\
 \hline 

\end{tabular}
\caption{Mit allem}
\end{center}
\end{table}
%-----------------------------------------------
\begin{table}
\begin{center}
\begin{tabular}{|l|r|c|}
 \hline
   linksb�ndig & rechtsb�ndig & zentriert \\
 \hline  
   123         &  123         & 123 \\
 \hline
   eins        &  zwei        & drei \\
 \hline 

\end{tabular}
\caption{Mit allem}
\end{center}
\end{table}

%-----------------------------------------------
\begin{table}
\begin{center}
\begin{tabular}{|l|r|c|}
 \hline
   linksb�ndig & rechtsb�ndig & zentriert \\
 \hline  
   123         &  123         & 123 \\
 \hline
   eins        &  zwei        & drei \\
 \hline 

\end{tabular}
\caption{Mit allem}
\end{center}
\end{table}
%-----------------------------------------------
\begin{table}
\begin{center}
\begin{tabular}{|l|r|c|}
 \hline
   linksb�ndig & rechtsb�ndig & zentriert \\
 \hline  
   123         &  123         & 123 \\
 \hline
   eins        &  zwei        & drei \\
 \hline 

\end{tabular}
\caption{Mit allem}
\end{center}
\end{table}

%-----------------------------------------------
\begin{table}
\begin{center}
\begin{tabular}{|l|r|c|}
 \hline
   linksb�ndig & rechtsb�ndig & zentriert \\
 \hline  
   123         &  123         & 123 \\
 \hline
   eins        &  zwei        & drei \\
 \hline 

\end{tabular}
\caption{Mit allem}
\end{center}
\end{table}
%-----------------------------------------------
\begin{table}
\begin{center}
\begin{tabular}{|l|r|c|}
 \hline
   linksb�ndig & rechtsb�ndig & zentriert \\
 \hline  
   123         &  123         & 123 \\
 \hline
   eins        &  zwei        & drei \\
 \hline 

\end{tabular}
\caption{Mit allem}
\end{center}
\end{table}

%-----------------------------------------------
\begin{table}
\begin{center}
\begin{tabular}{|l|r|c|}
 \hline
   linksb�ndig & rechtsb�ndig & zentriert \\
 \hline  
   123         &  123         & 123 \\
 \hline
   eins        &  zwei        & drei \\
 \hline 

\end{tabular}
\caption{Mit allem}
\end{center}
\end{table}
%-----------------------------------------------
\begin{table}
\begin{center}
\begin{tabular}{|l|r|c|}
 \hline
   linksb�ndig & rechtsb�ndig & zentriert \\
 \hline  
   123         &  123         & 123 \\
 \hline
   eins        &  zwei        & drei \\
 \hline 

\end{tabular}
\caption{Mit allem}
\end{center}
\end{table}


\clearpage


\begin{table}
\begin{center}
\begin{tabular}{|l|r|c|}
 \hline
   linksb�ndig & rechtsb�ndig & zentriert \\
 \hline  
   123         &  123         & 123 \\
 \hline
   eins        &  zwei        & drei \\
 \hline 

\end{tabular}
\caption{Mit allem}
\end{center}
\end{table}

%-----------------------------------------------
\begin{table}
\begin{center}
\begin{tabular}{|l|r|c|}
 \hline
   linksb�ndig & rechtsb�ndig & zentriert \\
 \hline  
   123         &  123         & 123 \\
 \hline
   eins        &  zwei        & drei \\
 \hline 

\end{tabular}
\caption{Mit allem}
\end{center}
\end{table}
%-----------------------------------------------
\begin{table}
\begin{center}
\begin{tabular}{|l|r|c|}
 \hline
   linksb�ndig & rechtsb�ndig & zentriert \\
 \hline  
   123         &  123         & 123 \\
 \hline
   eins        &  zwei        & drei \\
 \hline 

\end{tabular}
\caption{Mit allem}
\end{center}
\end{table}

%-----------------------------------------------
\begin{table}
\begin{center}
\begin{tabular}{|l|r|c|}
 \hline
   linksb�ndig & rechtsb�ndig & zentriert \\
 \hline  
   123         &  123         & 123 \\
 \hline
   eins        &  zwei        & drei \\
 \hline 

\end{tabular}
\caption{Mit allem}
\end{center}
\end{table}
%-----------------------------------------------
\begin{table}
\begin{center}
\begin{tabular}{|l|r|c|}
 \hline
   linksb�ndig & rechtsb�ndig & zentriert \\
 \hline  
   123         &  123         & 123 \\
 \hline
   eins        &  zwei        & drei \\
 \hline 

\end{tabular}
\caption{Mit allem}
\end{center}
\end{table}

%-----------------------------------------------
\begin{table}
\begin{center}
\begin{tabular}{|l|r|c|}
 \hline
   linksb�ndig & rechtsb�ndig & zentriert \\
 \hline  
   123         &  123         & 123 \\
 \hline
   eins        &  zwei        & drei \\
 \hline 

\end{tabular}
\caption{Mit allem}
\end{center}
\end{table}
%-----------------------------------------------
\begin{table}
\begin{center}
\begin{tabular}{|l|r|c|}
 \hline
   linksb�ndig & rechtsb�ndig & zentriert \\
 \hline  
   123         &  123         & 123 \\
 \hline
   eins        &  zwei        & drei \\
 \hline 

\end{tabular}
\caption{Mit allem}
\end{center}
\end{table}

%-----------------------------------------------
\begin{table}
\begin{center}
\begin{tabular}{|l|r|c|}
 \hline
   linksb�ndig & rechtsb�ndig & zentriert \\
 \hline  
   123         &  123         & 123 \\
 \hline
   eins        &  zwei        & drei \\
 \hline 

\end{tabular}
\caption{Mit allem}
\end{center}
\end{table}
%-----------------------------------------------
\begin{table}
\begin{center}
\begin{tabular}{|l|r|c|}
 \hline
   linksb�ndig & rechtsb�ndig & zentriert \\
 \hline  
   123         &  123         & 123 \\
 \hline
   eins        &  zwei        & drei \\
 \hline 

\end{tabular}
\caption{Mit allem}
\end{center}
\end{table}

%-----------------------------------------------
\begin{table}
\begin{center}
\begin{tabular}{|l|r|c|}
 \hline
   linksb�ndig & rechtsb�ndig & zentriert \\
 \hline  
   123         &  123         & 123 \\
 \hline
   eins        &  zwei        & drei \\
 \hline 

\end{tabular}
\caption{Mit allem}
\end{center}
\end{table}
%-----------------------------------------------
\begin{table}
\begin{center}
\begin{tabular}{|l|r|c|}
 \hline
   linksb�ndig & rechtsb�ndig & zentriert \\
 \hline  
   123         &  123         & 123 \\
 \hline
   eins        &  zwei        & drei \\
 \hline 

\end{tabular}
\caption{Mit allem}
\end{center}
\end{table}

%-----------------------------------------------
\begin{table}
\begin{center}
\begin{tabular}{|l|r|c|}
 \hline
   linksb�ndig & rechtsb�ndig & zentriert \\
 \hline  
   123         &  123         & 123 \\
 \hline
   eins        &  zwei        & drei \\
 \hline 

\end{tabular}
\caption{Mit allem}
\end{center}
\end{table}
%-----------------------------------------------
\begin{table}
\begin{center}
\begin{tabular}{|l|r|c|}
 \hline
   linksb�ndig & rechtsb�ndig & zentriert \\
 \hline  
   123         &  123         & 123 \\
 \hline
   eins        &  zwei        & drei \\
 \hline 

\end{tabular}
\caption{Mit allem}
\end{center}
\end{table}

%-----------------------------------------------
\begin{table}
\begin{center}
\begin{tabular}{|l|r|c|}
 \hline
   linksb�ndig & rechtsb�ndig & zentriert \\
 \hline  
   123         &  123         & 123 \\
 \hline
   eins        &  zwei        & drei \\
 \hline 

\end{tabular}
\caption{Mit allem}
\end{center}
\end{table}
%-----------------------------------------------
\begin{table}
\begin{center}
\begin{tabular}{|l|r|c|}
 \hline
   linksb�ndig & rechtsb�ndig & zentriert \\
 \hline  
   123         &  123         & 123 \\
 \hline
   eins        &  zwei        & drei \\
 \hline 

\end{tabular}
\caption{Mit allem}
\end{center}
\end{table}


\clearpage

\begin{table}
\begin{center}
\begin{tabular}{|l|r|c|}
 \hline
   linksb�ndig & rechtsb�ndig & zentriert \\
 \hline  
   123         &  123         & 123 \\
 \hline
   eins        &  zwei        & drei \\
 \hline 

\end{tabular}
\caption{Mit allem}
\end{center}
\end{table}

%-----------------------------------------------
\begin{table}
\begin{center}
\begin{tabular}{|l|r|c|}
 \hline
   linksb�ndig & rechtsb�ndig & zentriert \\
 \hline  
   123         &  123         & 123 \\
 \hline
   eins        &  zwei        & drei \\
 \hline 

\end{tabular}
\caption{Mit allem}
\end{center}
\end{table}
%-----------------------------------------------
\begin{table}
\begin{center}
\begin{tabular}{|l|r|c|}
 \hline
   linksb�ndig & rechtsb�ndig & zentriert \\
 \hline  
   123         &  123         & 123 \\
 \hline
   eins        &  zwei        & drei \\
 \hline 

\end{tabular}
\caption{Mit allem}
\end{center}
\end{table}

%-----------------------------------------------
\begin{table}
\begin{center}
\begin{tabular}{|l|r|c|}
 \hline
   linksb�ndig & rechtsb�ndig & zentriert \\
 \hline  
   123         &  123         & 123 \\
 \hline
   eins        &  zwei        & drei \\
 \hline 

\end{tabular}
\caption{Mit allem}
\end{center}
\end{table}
%-----------------------------------------------
\begin{table}
\begin{center}
\begin{tabular}{|l|r|c|}
 \hline
   linksb�ndig & rechtsb�ndig & zentriert \\
 \hline  
   123         &  123         & 123 \\
 \hline
   eins        &  zwei        & drei \\
 \hline 

\end{tabular}
\caption{Mit allem}
\end{center}
\end{table}

%-----------------------------------------------
\begin{table}
\begin{center}
\begin{tabular}{|l|r|c|}
 \hline
   linksb�ndig & rechtsb�ndig & zentriert \\
 \hline  
   123         &  123         & 123 \\
 \hline
   eins        &  zwei        & drei \\
 \hline 

\end{tabular}
\caption{Mit allem}
\end{center}
\end{table}
%-----------------------------------------------
\begin{table}
\begin{center}
\begin{tabular}{|l|r|c|}
 \hline
   linksb�ndig & rechtsb�ndig & zentriert \\
 \hline  
   123         &  123         & 123 \\
 \hline
   eins        &  zwei        & drei \\
 \hline 

\end{tabular}
\caption{Mit allem}
\end{center}
\end{table}

%-----------------------------------------------
\begin{table}
\begin{center}
\begin{tabular}{|l|r|c|}
 \hline
   linksb�ndig & rechtsb�ndig & zentriert \\
 \hline  
   123         &  123         & 123 \\
 \hline
   eins        &  zwei        & drei \\
 \hline 

\end{tabular}
\caption{Mit allem}
\end{center}
\end{table}
%-----------------------------------------------
\begin{table}
\begin{center}
\begin{tabular}{|l|r|c|}
 \hline
   linksb�ndig & rechtsb�ndig & zentriert \\
 \hline  
   123         &  123         & 123 \\
 \hline
   eins        &  zwei        & drei \\
 \hline 

\end{tabular}
\caption{Mit allem}
\end{center}
\end{table}

%-----------------------------------------------
\begin{table}
\begin{center}
\begin{tabular}{|l|r|c|}
 \hline
   linksb�ndig & rechtsb�ndig & zentriert \\
 \hline  
   123         &  123         & 123 \\
 \hline
   eins        &  zwei        & drei \\
 \hline 

\end{tabular}
\caption{Mit allem}
\end{center}
\end{table}
%-----------------------------------------------
\begin{table}
\begin{center}
\begin{tabular}{|l|r|c|}
 \hline
   linksb�ndig & rechtsb�ndig & zentriert \\
 \hline  
   123         &  123         & 123 \\
 \hline
   eins        &  zwei        & drei \\
 \hline 

\end{tabular}
\caption{Mit allem}
\end{center}
\end{table}

%-----------------------------------------------
\begin{table}
\begin{center}
\begin{tabular}{|l|r|c|}
 \hline
   linksb�ndig & rechtsb�ndig & zentriert \\
 \hline  
   123         &  123         & 123 \\
 \hline
   eins        &  zwei        & drei \\
 \hline 

\end{tabular}
\caption{Mit allem}
\end{center}
\end{table}
%-----------------------------------------------
\begin{table}
\begin{center}
\begin{tabular}{|l|r|c|}
 \hline
   linksb�ndig & rechtsb�ndig & zentriert \\
 \hline  
   123         &  123         & 123 \\
 \hline
   eins        &  zwei        & drei \\
 \hline 

\end{tabular}
\caption{Mit allem}
\end{center}
\end{table}

%-----------------------------------------------
\begin{table}
\begin{center}
\begin{tabular}{|l|r|c|}
 \hline
   linksb�ndig & rechtsb�ndig & zentriert \\
 \hline  
   123         &  123         & 123 \\
 \hline
   eins        &  zwei        & drei \\
 \hline 

\end{tabular}
\caption{Mit allem}
\end{center}
\end{table}
%-----------------------------------------------
\begin{table}
\begin{center}
\begin{tabular}{|l|r|c|}
 \hline
   linksb�ndig & rechtsb�ndig & zentriert \\
 \hline  
   123         &  123         & 123 \\
 \hline
   eins        &  zwei        & drei \\
 \hline 

\end{tabular}
\caption{Mit allem}
\end{center}
\end{table}


\clearpage

%-----------------------------------------------
\begin{table}
\begin{center}
\begin{tabular}{|l|r|c|}
 \hline
   linksb�ndig & rechtsb�ndig & zentriert \\
 \hline  
   123         &  123         & 123 \\
 \hline
   eins        &  zwei        & drei \\
 \hline 

\end{tabular}
\caption{Mit allem}
\end{center}
\end{table}

%-----------------------------------------------
\begin{table}
\begin{center}
\begin{tabular}{|l|r|c|}
 \hline
   linksb�ndig & rechtsb�ndig & zentriert \\
 \hline  
   123         &  123         & 123 \\
 \hline
   eins        &  zwei        & drei \\
 \hline 

\end{tabular}
\caption{Mit allem}
\end{center}
\end{table}
%-----------------------------------------------
\begin{table}
\begin{center}
\begin{tabular}{|l|r|c|}
 \hline
   linksb�ndig & rechtsb�ndig & zentriert \\
 \hline  
   123         &  123         & 123 \\
 \hline
   eins        &  zwei        & drei \\
 \hline 

\end{tabular}
\caption{Mit allem}
\end{center}
\end{table}

%-----------------------------------------------
\begin{table}
\begin{center}
\begin{tabular}{|l|r|c|}
 \hline
   linksb�ndig & rechtsb�ndig & zentriert \\
 \hline  
   123         &  123         & 123 \\
 \hline
   eins        &  zwei        & drei \\
 \hline 

\end{tabular}
\caption{Mit allem}
\end{center}
\end{table}
%-----------------------------------------------
\begin{table}
\begin{center}
\begin{tabular}{|l|r|c|}
 \hline
   linksb�ndig & rechtsb�ndig & zentriert \\
 \hline  
   123         &  123         & 123 \\
 \hline
   eins        &  zwei        & drei \\
 \hline 

\end{tabular}
\caption{Mit allem}
\end{center}
\end{table}

%-----------------------------------------------
\begin{table}
\begin{center}
\begin{tabular}{|l|r|c|}
 \hline
   linksb�ndig & rechtsb�ndig & zentriert \\
 \hline  
   123         &  123         & 123 \\
 \hline
   eins        &  zwei        & drei \\
 \hline 

\end{tabular}
\caption{Mit allem}
\end{center}
\end{table}
%-----------------------------------------------
\begin{table}
\begin{center}
\begin{tabular}{|l|r|c|}
 \hline
   linksb�ndig & rechtsb�ndig & zentriert \\
 \hline  
   123         &  123         & 123 \\
 \hline
   eins        &  zwei        & drei \\
 \hline 

\end{tabular}
\caption{Mit allem}
\end{center}
\end{table}

%-----------------------------------------------
\begin{table}
\begin{center}
\begin{tabular}{|l|r|c|}
 \hline
   linksb�ndig & rechtsb�ndig & zentriert \\
 \hline  
   123         &  123         & 123 \\
 \hline
   eins        &  zwei        & drei \\
 \hline 

\end{tabular}
\caption{Mit allem}
\end{center}
\end{table}
%-----------------------------------------------
\begin{table}
\begin{center}
\begin{tabular}{|l|r|c|}
 \hline
   linksb�ndig & rechtsb�ndig & zentriert \\
 \hline  
   123         &  123         & 123 \\
 \hline
   eins        &  zwei        & drei \\
 \hline 

\end{tabular}
\caption{Mit allem}
\end{center}
\end{table}

%-----------------------------------------------
\begin{table}
\begin{center}
\begin{tabular}{|l|r|c|}
 \hline
   linksb�ndig & rechtsb�ndig & zentriert \\
 \hline  
   123         &  123         & 123 \\
 \hline
   eins        &  zwei        & drei \\
 \hline 

\end{tabular}
\caption{Mit allem}
\end{center}
\end{table}
%-----------------------------------------------
\begin{table}
\begin{center}
\begin{tabular}{|l|r|c|}
 \hline
   linksb�ndig & rechtsb�ndig & zentriert \\
 \hline  
   123         &  123         & 123 \\
 \hline
   eins        &  zwei        & drei \\
 \hline 

\end{tabular}
\caption{Mit allem}
\end{center}
\end{table}

%-----------------------------------------------
\begin{table}
\begin{center}
\begin{tabular}{|l|r|c|}
 \hline
   linksb�ndig & rechtsb�ndig & zentriert \\
 \hline  
   123         &  123         & 123 \\
 \hline
   eins        &  zwei        & drei \\
 \hline 

\end{tabular}
\caption{Mit allem}
\end{center}
\end{table}
%-----------------------------------------------
\begin{table}
\begin{center}
\begin{tabular}{|l|r|c|}
 \hline
   linksb�ndig & rechtsb�ndig & zentriert \\
 \hline  
   123         &  123         & 123 \\
 \hline
   eins        &  zwei        & drei \\
 \hline 

\end{tabular}
\caption{Mit allem}
\end{center}
\end{table}

%-----------------------------------------------
\begin{table}
\begin{center}
\begin{tabular}{|l|r|c|}
 \hline
   linksb�ndig & rechtsb�ndig & zentriert \\
 \hline  
   123         &  123         & 123 \\
 \hline
   eins        &  zwei        & drei \\
 \hline 

\end{tabular}
\caption{Mit allem}
\end{center}
\end{table}
%-----------------------------------------------
\begin{table}
\begin{center}
\begin{tabular}{|l|r|c|}
 \hline
   linksb�ndig & rechtsb�ndig & zentriert \\
 \hline  
   123         &  123         & 123 \\
 \hline
   eins        &  zwei        & drei \\
 \hline 

\end{tabular}
\caption{Mit allem}
\end{center}
\end{table}


\clearpage


\begin{table}
\begin{center}
\begin{tabular}{|l|r|c|}
 \hline
   linksb�ndig & rechtsb�ndig & zentriert \\
 \hline  
   123         &  123         & 123 \\
 \hline
   eins        &  zwei        & drei \\
 \hline 

\end{tabular}
\caption{Mit allem}
\end{center}
\end{table}

%-----------------------------------------------
\begin{table}
\begin{center}
\begin{tabular}{|l|r|c|}
 \hline
   linksb�ndig & rechtsb�ndig & zentriert \\
 \hline  
   123         &  123         & 123 \\
 \hline
   eins        &  zwei        & drei \\
 \hline 

\end{tabular}
\caption{Mit allem}
\end{center}
\end{table}
%-----------------------------------------------
\begin{table}
\begin{center}
\begin{tabular}{|l|r|c|}
 \hline
   linksb�ndig & rechtsb�ndig & zentriert \\
 \hline  
   123         &  123         & 123 \\
 \hline
   eins        &  zwei        & drei \\
 \hline 

\end{tabular}
\caption{Mit allem}
\end{center}
\end{table}

%-----------------------------------------------
\begin{table}
\begin{center}
\begin{tabular}{|l|r|c|}
 \hline
   linksb�ndig & rechtsb�ndig & zentriert \\
 \hline  
   123         &  123         & 123 \\
 \hline
   eins        &  zwei        & drei \\
 \hline 

\end{tabular}
\caption{Mit allem}
\end{center}
\end{table}
%-----------------------------------------------
\begin{table}
\begin{center}
\begin{tabular}{|l|r|c|}
 \hline
   linksb�ndig & rechtsb�ndig & zentriert \\
 \hline  
   123         &  123         & 123 \\
 \hline
   eins        &  zwei        & drei \\
 \hline 

\end{tabular}
\caption{Mit allem}
\end{center}
\end{table}

%-----------------------------------------------
\begin{table}
\begin{center}
\begin{tabular}{|l|r|c|}
 \hline
   linksb�ndig & rechtsb�ndig & zentriert \\
 \hline  
   123         &  123         & 123 \\
 \hline
   eins        &  zwei        & drei \\
 \hline 

\end{tabular}
\caption{Mit allem}
\end{center}
\end{table}
%-----------------------------------------------
\begin{table}
\begin{center}
\begin{tabular}{|l|r|c|}
 \hline
   linksb�ndig & rechtsb�ndig & zentriert \\
 \hline  
   123         &  123         & 123 \\
 \hline
   eins        &  zwei        & drei \\
 \hline 

\end{tabular}
\caption{Mit allem}
\end{center}
\end{table}

%-----------------------------------------------
\begin{table}
\begin{center}
\begin{tabular}{|l|r|c|}
 \hline
   linksb�ndig & rechtsb�ndig & zentriert \\
 \hline  
   123         &  123         & 123 \\
 \hline
   eins        &  zwei        & drei \\
 \hline 

\end{tabular}
\caption{Mit allem}
\end{center}
\end{table}
%-----------------------------------------------
\begin{table}
\begin{center}
\begin{tabular}{|l|r|c|}
 \hline
   linksb�ndig & rechtsb�ndig & zentriert \\
 \hline  
   123         &  123         & 123 \\
 \hline
   eins        &  zwei        & drei \\
 \hline 

\end{tabular}
\caption{Mit allem}
\end{center}
\end{table}

%-----------------------------------------------
\begin{table}
\begin{center}
\begin{tabular}{|l|r|c|}
 \hline
   linksb�ndig & rechtsb�ndig & zentriert \\
 \hline  
   123         &  123         & 123 \\
 \hline
   eins        &  zwei        & drei \\
 \hline 

\end{tabular}
\caption{Mit allem}
\end{center}
\end{table}
%-----------------------------------------------
\begin{table}
\begin{center}
\begin{tabular}{|l|r|c|}
 \hline
   linksb�ndig & rechtsb�ndig & zentriert \\
 \hline  
   123         &  123         & 123 \\
 \hline
   eins        &  zwei        & drei \\
 \hline 

\end{tabular}
\caption{Mit allem}
\end{center}
\end{table}

%-----------------------------------------------
\begin{table}
\begin{center}
\begin{tabular}{|l|r|c|}
 \hline
   linksb�ndig & rechtsb�ndig & zentriert \\
 \hline  
   123         &  123         & 123 \\
 \hline
   eins        &  zwei        & drei \\
 \hline 

\end{tabular}
\caption{Mit allem}
\end{center}
\end{table}
%-----------------------------------------------
\begin{table}
\begin{center}
\begin{tabular}{|l|r|c|}
 \hline
   linksb�ndig & rechtsb�ndig & zentriert \\
 \hline  
   123         &  123         & 123 \\
 \hline
   eins        &  zwei        & drei \\
 \hline 

\end{tabular}
\caption{Mit allem}
\end{center}
\end{table}

%-----------------------------------------------
\begin{table}
\begin{center}
\begin{tabular}{|l|r|c|}
 \hline
   linksb�ndig & rechtsb�ndig & zentriert \\
 \hline  
   123         &  123         & 123 \\
 \hline
   eins        &  zwei        & drei \\
 \hline 

\end{tabular}
\caption{Mit allem}
\end{center}
\end{table}
%-----------------------------------------------
\begin{table}
\begin{center}
\begin{tabular}{|l|r|c|}
 \hline
   linksb�ndig & rechtsb�ndig & zentriert \\
 \hline  
   123         &  123         & 123 \\
 \hline
   eins        &  zwei        & drei \\
 \hline 

\end{tabular}
\caption{Mit allem}
\end{center}
\end{table}


\clearpage

\begin{table}
\begin{center}
\begin{tabular}{|l|r|c|}
 \hline
   linksb�ndig & rechtsb�ndig & zentriert \\
 \hline  
   123         &  123         & 123 \\
 \hline
   eins        &  zwei        & drei \\
 \hline 

\end{tabular}
\caption{Mit allem}
\end{center}
\end{table}

%-----------------------------------------------
\begin{table}
\begin{center}
\begin{tabular}{|l|r|c|}
 \hline
   linksb�ndig & rechtsb�ndig & zentriert \\
 \hline  
   123         &  123         & 123 \\
 \hline
   eins        &  zwei        & drei \\
 \hline 

\end{tabular}
\caption{Mit allem}
\end{center}
\end{table}
%-----------------------------------------------
\begin{table}
\begin{center}
\begin{tabular}{|l|r|c|}
 \hline
   linksb�ndig & rechtsb�ndig & zentriert \\
 \hline  
   123         &  123         & 123 \\
 \hline
   eins        &  zwei        & drei \\
 \hline 

\end{tabular}
\caption{Mit allem}
\end{center}
\end{table}

%-----------------------------------------------
\begin{table}
\begin{center}
\begin{tabular}{|l|r|c|}
 \hline
   linksb�ndig & rechtsb�ndig & zentriert \\
 \hline  
   123         &  123         & 123 \\
 \hline
   eins        &  zwei        & drei \\
 \hline 

\end{tabular}
\caption{Mit allem}
\end{center}
\end{table}
%-----------------------------------------------
\begin{table}
\begin{center}
\begin{tabular}{|l|r|c|}
 \hline
   linksb�ndig & rechtsb�ndig & zentriert \\
 \hline  
   123         &  123         & 123 \\
 \hline
   eins        &  zwei        & drei \\
 \hline 

\end{tabular}
\caption{Mit allem}
\end{center}
\end{table}

%-----------------------------------------------
\begin{table}
\begin{center}
\begin{tabular}{|l|r|c|}
 \hline
   linksb�ndig & rechtsb�ndig & zentriert \\
 \hline  
   123         &  123         & 123 \\
 \hline
   eins        &  zwei        & drei \\
 \hline 

\end{tabular}
\caption{Mit allem}
\end{center}
\end{table}
%-----------------------------------------------
\begin{table}
\begin{center}
\begin{tabular}{|l|r|c|}
 \hline
   linksb�ndig & rechtsb�ndig & zentriert \\
 \hline  
   123         &  123         & 123 \\
 \hline
   eins        &  zwei        & drei \\
 \hline 

\end{tabular}
\caption{Mit allem}
\end{center}
\end{table}

%-----------------------------------------------
\begin{table}
\begin{center}
\begin{tabular}{|l|r|c|}
 \hline
   linksb�ndig & rechtsb�ndig & zentriert \\
 \hline  
   123         &  123         & 123 \\
 \hline
   eins        &  zwei        & drei \\
 \hline 

\end{tabular}
\caption{Mit allem}
\end{center}
\end{table}
%-----------------------------------------------
\begin{table}
\begin{center}
\begin{tabular}{|l|r|c|}
 \hline
   linksb�ndig & rechtsb�ndig & zentriert \\
 \hline  
   123         &  123         & 123 \\
 \hline
   eins        &  zwei        & drei \\
 \hline 

\end{tabular}
\caption{Mit allem}
\end{center}
\end{table}

%-----------------------------------------------
\begin{table}
\begin{center}
\begin{tabular}{|l|r|c|}
 \hline
   linksb�ndig & rechtsb�ndig & zentriert \\
 \hline  
   123         &  123         & 123 \\
 \hline
   eins        &  zwei        & drei \\
 \hline 

\end{tabular}
\caption{Mit allem}
\end{center}
\end{table}
%-----------------------------------------------
\begin{table}
\begin{center}
\begin{tabular}{|l|r|c|}
 \hline
   linksb�ndig & rechtsb�ndig & zentriert \\
 \hline  
   123         &  123         & 123 \\
 \hline
   eins        &  zwei        & drei \\
 \hline 

\end{tabular}
\caption{Mit allem}
\end{center}
\end{table}

%-----------------------------------------------
\begin{table}
\begin{center}
\begin{tabular}{|l|r|c|}
 \hline
   linksb�ndig & rechtsb�ndig & zentriert \\
 \hline  
   123         &  123         & 123 \\
 \hline
   eins        &  zwei        & drei \\
 \hline 

\end{tabular}
\caption{Mit allem}
\end{center}
\end{table}
%-----------------------------------------------
\begin{table}
\begin{center}
\begin{tabular}{|l|r|c|}
 \hline
   linksb�ndig & rechtsb�ndig & zentriert \\
 \hline  
   123         &  123         & 123 \\
 \hline
   eins        &  zwei        & drei \\
 \hline 

\end{tabular}
\caption{Mit allem}
\end{center}
\end{table}

%-----------------------------------------------
\begin{table}
\begin{center}
\begin{tabular}{|l|r|c|}
 \hline
   linksb�ndig & rechtsb�ndig & zentriert \\
 \hline  
   123         &  123         & 123 \\
 \hline
   eins        &  zwei        & drei \\
 \hline 

\end{tabular}
\caption{Mit allem}
\end{center}
\end{table}
%-----------------------------------------------
\begin{table}
\begin{center}
\begin{tabular}{|l|r|c|}
 \hline
   linksb�ndig & rechtsb�ndig & zentriert \\
 \hline  
   123         &  123         & 123 \\
 \hline
   eins        &  zwei        & drei \\
 \hline 

\end{tabular}
\caption{Mit allem}
\end{center}
\end{table}


\clearpage

  % Test f�r langes Tabellenverzeichnis

