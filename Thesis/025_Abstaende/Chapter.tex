%%
%% Kapitel:
%%
\chapter{Abst�nde}
\label{sec:Abstaende}
%%======================================================================

Abst�nde, Einz�ge, Zentrierungen und so weiter.

\section{Umbruch}
%
Einen Zeilenumbruch wird  mit \verb=\\= erzeugt. \\ 
Alternativ kann eine neue Zeile mit
\verb# \newline # angefangen werden.

Ein neuer Absatz entsteht durch eine Leerzeile im Quelltext.
Leerzeilen m�ssen also mit \verb#%# beginnen, wenn
kein neuer Absatz beginnen soll.

Eine neue Seite wird mit \verb=\newpage= erzeugt. 


\section{Abst�nde}
%
\subsection{Zeilenabstand}
Der folgende Absatz verdeutlicht, dass einzeiliger Text nicht immer ausreichend ist.


\begin{spacing}{1.0}.
%\linespread{1,0}          % Zeilenabstand erst nach dem Titel setzen !
Das Model von Chisholm bezieht den Zweiphasenmultiplikator $\phi^{2}\idx{f,g}$ auf die fl�ssige Phase. Zur Berechnung von $\Delta p\idx{f}$ wird angenommen, dass die Fl�ssigkeit allein im Rohr str�mt und den Querschnitt vollst�ndig ausf�llt. Das Grundmodell beschreibt, dass man den zweiphasigen Druckabfall $\Delta p\idx{R}$ mit dem Druckabfall einer Phase $\Delta p\idx{f,g}$ und mit einem Zweiphasenmultiplikator $\phi^{2}\idx{f,g}$ berechnen kann.
\end{spacing}

Ein 1,2-zeiliger Abstand im ganzen Dokument kann Abhilfe schaffen.

\begin{spacing}{1.2}
Das Model von Chisholm bezieht den Zweiphasenmultiplikator $\phi^{2}\idx{f,g}$ auf die fl�ssige Phase. Zur Berechnung von $\Delta p\idx{f}$ wird angenommen, dass die Fl�ssigkeit allein im Rohr str�mt und den Querschnitt vollst�ndig ausf�llt. Das Grundmodell beschreibt, dass man den zweiphasigen Druckabfall $\Delta p\idx{R}$ mit dem Druckabfall einer Phase $\Delta p\idx{f,g}$ und mit einem Zweiphasenmultiplikator $\phi^{2}\idx{f,g}$ berechnen kann.
\end{spacing}
  

\subsection{Ma�einheiten}
%
{\LaTeX} kennt die �blichen Ma�einheiten wie beispielsweise mm, cm, pt. 
Aber besser sind f�r Breiten die \emph{em}, was der Breite eines gro�en
M in der Basisschrift entspricht. F�r H�hen wird \emph{ex} genommen, 
was der H�he eines gro�en X entspricht.
%
%
\subsection{Text schieben}
%
Mit \zB \verb=\hspace{5em}= wird horizontal \hspace{5em} verschoben. 
\verb=\hfill= verschiebt den Text  an den \hfill rechten \hfill Rand.%

Jetzt folgt ein \verb=\hfill{12ex}= 
\vspace{12ex}
um den Text weiter unten fortzusetzen.
\verb=\vfill= f�llt die Seite (siehe R�ckseite der Titelseite). 


\section{Zentrierter Text}

\centerline{So wird eine Zeile zentriert.}

\begin{center}
Hier ist ein zentrierter Absatz:\\
Franz jagt im komplett verwahrlosten Taxi quer durch Bayern.
Franz jagt im komplett verwahrlosten Taxi quer durch Bayern.
\end{center}



\section{Eingezogener Text}

\marginline{\rule[-20mm]{2mm}{20mm}} %Anfangspunkt, Dicke, H�he
%\marginpar[lange linke Luft-Notiz]{rechte Dampfschifffahrtsgesellschaftskapit�n 
%rechte rechte rechte rechte rechte rechte rechte rechte }
%
\marginline{lange flinke Luft-Notiz}
Franz jagt im komplett verwahrlosten Taxi quer durch Bayern
und merkt sich das mit einem Balken.

\begin{quote}
   \verb#\begin{quote}# wird unter anderem f�r Zitate benutzt.
   Franz jagt im komplett verwahrlosten Taxi quer durch Bayern.
   Franz jagt im komplett verwahrlosten Taxi quer durch Bayern.
\end{quote}
Franz jagt im komplett verwahrlosten Taxi quer durch Bayern.
Franz jagt im komplett verwahrlosten Taxi quer durch Bayern.



